% !TEX root = ./informe.tex

\section{Greedy}

\subsection{Explicación}
Teniendo un conjunto solución $S$ que forme una clique, golosamente podemos agregar a $S$ algún nodo que haga aumentar el tamaño de la frontera. \\

Mas detalladamente, el algoritmo funciona de la siguiente manera: \\

Primero considero que todos los nodos son candidatos, y voy a ejecutar el algoritmo siempre que exista algún candidato en la lista. Además mantengo un vector $S$ que representa mi solución actual, inicialmente vacío. Considero que mi frontera máxima $FM$ tiene valor $-1$. \\

Para cada iteracion del ciclo principal, quiero agregar el mejor candidato posible a mi solución. Esto significa iterar por todos mis candidatos y agregar a la solución aquel que maximice la frontera. \\

Recorro mi lista de candidatos y tomo alguno, $c$. Considero $c$ como parte de la solución. Si la solución $S$ no forma una clique, entonces quito $c$ de $S$ , considero que ya no es mas un candidato y vuelvo al comienzo del algortimo. Si $S$ forma una clique, entonces calculo su frontera $f$. \\

Comparo la frontera $f$ con la frontera máxima $FM$. Si $f < FM$, entonces quito a $c$ de $S$, pues no hace que la solucion mejore. Si $f \geq FM$, entonces mantengo a $c$, entonces $c$ es mi mejor candidato, pues la frontera máxima no empeoró. Actualizo $FM$ con el valor de $f$. En ambos casos quito tambíen a $c$ de la lista de candidatos, para no repetirlo dos veces. \\

Una vez encontrado el $c$ que maximice mi solución, lo agrego definitivamente y sigo iterando hasta que ya no pueda considerar mas candidatos. \\

Finalizado el algoritmo, en $S$ va a haber una clique elegida de manera golosa, con la mayor frontera que se puedo encontrar. \\


\subsection{Pseudocódigo}
\todo[inline]{Pseudocodigo}

Referencias de variables globales para el pseudocódigo:
\begin{itemize}
    \item $n$: La cantidad de nodos
    \item $solucion$: Secuencia que contiene la clique solución
    \item $candidatos$: Secuencia con los nodos a considerar
    \item $fronteraMax$: El cardinal de la frontera de la clique solución
\end{itemize}


\begin{algorithm}[H]
\begin{algorithmic}
\Function{Resolver}{}
    \State LeerInput() \Comment $O(n^2)$
    \State $solucion \gets \emptyset$ \Comment $O(1)$
    \State $candidatos \gets \{1, 2, 3, .. , n\} $ \Comment $O(n)$
    \State $fronteraMax \gets -1$  \Comment $O(1)$
    \State ObtenerSolucion()
\EndFunction
\end{algorithmic}
\end{algorithm}



\begin{algorithm}[H]
\begin{algorithmic}
\Function{ObtenerSolucion}{}

    \While{$candidatos \neq \emptyset$}

        \State $mejorCandidato \gets -1$

        \For{$c \in candidatos$}
            \State $solucion.push(c)$ \Comment $O(1)$

            \If {$\neg$EsClique($solucion$)}
                \State $solucion.erase(c)$
                \State $candidatos.erase(c)$
            \Else
                \If{$fronteraActual \geq fronteraMax$}
                    \State $fronteraMax \gets fronteraActual$
                    \State $mejorCandidato \gets c$
                \Else
                    \State $candidatos.erase(c)$
                \EndIf

                \State $solucion.erase(c)$
            \EndIf
        \EndFor

        \If{$mejorCandidato \neq -1$}
            \State $solucion.push(mejorCandidato)$ \Comment $O(1)$
            \State $candidatos.erase(mejorCandidato)$

        \EndIf
    \EndWhile

    \State return $solucion$

\EndFunction

\end{algorithmic}
\end{algorithm}


\subsection{Complejidad}
\todo[inline]{Dar complejidad y ver por que es mucho mejor que el exacto}

\subsection{Optimalidad}
\todo[inline]{Mostrar los casos donde no da la solucion optima y que pueden alejarse tanto como se quiera}

\subsection{Experimentación}
\todo[inline]{Se explica solo}
