-% !TEX root = ./informe.tex

\section{Algoritmo Exacto}

\subsection{Explicación}
Para saber si un conjunto $vs$ de vértices es clique, revisamos si todos los vértices en $vs$ son vecinos entre sí.
Esto significa que forman un subgrafo completo (son subgrafo pues los vértices y aristas forman parte del grafo principal), por lo tanto es clique. \\

Para calcular una frontera, dado una clique $c$ recorremos cada vértice $v$ de la clique y vemos cuales son los vecinos de $v$ que \textbf{no} están en $c$. \\

Como una clique está formada por nodos del grafo, para resolver el problema de encontrar la clique de frontera máxima (CFM) revisamos todos los posibles subconjuntos de nodos. En caso de que formen clique, calculamos su frontera y nos quedamos con el máximo de ellas. \\

Recorriendo todos los subconjuntos posibles de nodos estamos recorriendo todas las posibles cliques, y nos quedamos con aquella que tenga máxima frontera. Por lo tanto es un algortimo exacto, siempre encuentra la solución óptima. \\

\subsection{Pseudocódigo}

% (Ver notas debajo del pseudocodigo las referencia de significados de las variables)
Referencias de variables globales para el pseudocódigo:
\begin{itemize}
    \item $n$: La cantidad de nodos
    \item $solucion$: Secuencia que contiene la clique solución
    \item $fronteraMax$: El cardinal de la frontera de la clique solución
\end{itemize}

\begin{algorithm}[H]
\begin{algorithmic}
\Function{Resolver}{}
    \State LeerInput()                  \Comment $O(n^2)$
    \State $solucion \gets \emptyset$   \Comment $O(1)$
    \State $fronteraMax \gets 0$        \Comment $O(1)$
    \State GenerarSubconjuntos($\emptyset$, $0$) \Comment $O(2^{n} * n^{3})$
\EndFunction
\end{algorithmic}
\end{algorithm}

\begin{algorithm}[H]
\begin{algorithmic}
\Function{GenerarSubconjuntos}{$conjNodos$, $actual$} \Comment $O(2^{n} * n^{3})$, ver detalle en sección Complejidad.

    \If {$actual = n$}                  \Comment $O(1)$
        \If {EsClique($conjNodos$)}     \Comment $O(n^3)$
            \State $fronteraActual \gets$ Frontera($conjNodos$) \Comment $O(n^2)$
            \If {$fronteraActual > fronteraMax$}          \Comment $O(1)$
                \State $fronteraMax \gets fronteraActual$ \Comment $O(1)$
                \State $solucion \gets conjNodos$         \Comment $O(1)$
            \EndIf
        \EndIf

    \Else \Comment Ver sección Complejidad.
        \State GenerarSubconjuntos($conjNodos$, $actual + 1$)
        \State GenerarSubconjuntos($conjNodos \cup \{actual\}$, $actual + 1$)
    \EndIf
\EndFunction
\end{algorithmic}
\end{algorithm}

\begin{algorithm}[H]
\begin{algorithmic}
\Function{EsClique}{$conjNodos$}
    \For{$v \in conjNodos$}                     \Comment $O(n^3)$
        \For{$w \in conjNodos$}                 \Comment $O(n^2)$
            \If{$(v \neq w) \land (\neg$SonVecinos($v$, $w$)$)$}  \Comment $O(n)$
                \State return $False$                   \Comment $O(1)$
            \EndIf
        \EndFor
    \EndFor
    \State return $True$
\EndFunction
\end{algorithmic}
\end{algorithm}

\begin{algorithm}[H]
\begin{algorithmic}
\Function{SonVecinos}{$v1, v2$}
    \For{$w \in vecinos[v1]$}    \Comment $O(n)$
        \If{$w = v2$}            \Comment $O(1)$
            \State return $True$ \Comment $O(1)$
        \EndIf
    \EndFor
    \State return $False$        \Comment $O(1)$

\EndFunction
\end{algorithmic}
\end{algorithm}

\begin{algorithm}[H]
\begin{algorithmic}
\Function{Frontera}{$clique$}
    \State $enClique \gets vector(n, False)$ \Comment Vector con $n$ Falses, $O(n)$
    \For {$v \in clique$}                    \Comment $O(n)$
        \State $enClique[v] \gets True$      \Comment $O(1)$
    \EndFor

    \State $contador \gets 0$                \Comment $O(1)$
    \For{$v \in clique$}                     \Comment $O(n^2)$
        \For{$vecino \in vecinos[v]$}        \Comment $O(n)$
            \If{$\neg enClique[vecino]$}     \Comment $O(1)$
                \State $contador++$          \Comment $O(1)$
            \EndIf
        \EndFor
    \EndFor

    \State return $contador$                 \Comment $O(1)$

\EndFunction
\end{algorithmic}
\end{algorithm}

\subsection{Complejidad}

\begin{itemize}
    \item LeerInput es $O(n^2)$: Leo como máximo todas las aristas posibles.
    \item SonVecinos es $O(n)$: Recorro la lista de adyacencia de un vértice, como mucho tiene $O(n)$ vecinos.
    \item EsClique es $O(n^{3})$: Para cada vértice se recorren todos los demas y se revisa si son vecinos.
    \item Frontera es $O(n^{2})$: Para cada vértice de la clique se revisan todos sus vecinos.
    \item GenerarSubconjuntos es $O(2^{n} * n^{3})$: Hay $O(2^n)$ llamados recursivos (pues cada vertice está o no está), y en cada llamado recursivo se revisa si EsClique en $O(n^3)$ y luego su Frontera en $O(n^2)$ ($=$ $O(n^3)$). También puede pensarse que se generan $2^n$ subconjuntos, y por cada subconjunto se hacen $O(n^3)$ operaciones.
\end{itemize}

Para Resolver() se lee el input y luego se llama a GenerarSubconjuntos. La complejidad final es:

$$ O(n^2) + O(2^{n} * n^{3}) = O(2^{n} * n^{3})$$

\todo[inline]{Esto es demasiado lento bla veamos formas aproximadas de resolverlo blabla}