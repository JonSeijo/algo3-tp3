\section{Algoritmo Exacto}

\subsection{Explicación}
\todo[inline]{Aca explicar como funciona el algoritmo}
Basicamente genero todos los subconjuntos y si es clique me quedo con la frontera

\subsection{Pseudocódigo}
\todo[inline]{Completar el pseudocodigo}
\todo[inline]{Agregar las complejidades}

% (Ver notas debajo del pseudocodigo las referencia de significados de las variables)
Referencias de variables globales para el pseudocódigo:
\begin{itemize}
    \item $n$: La cantidad de nodos
    \item $solucion$: Secuencia que contiene la clique solución
    \item $fronteraMax$: El cardinal de la frontera de la clique solución
\end{itemize}

\begin{algorithm}[H]
\begin{algorithmic}

\Function{Resolver}{}
    \State LeerInput()
    \State $solucion \gets \emptyset$
    \State $fronteraMax \gets 0$
    \State GenerarSubconjuntos($\emptyset$, $0$)
\EndFunction

$ $\newline

\Function{GenerarSubconjuntos}{$conjNodos$, $actual$}
    \If {$actual = n$}

        \If {EsClique($conjNodos$)}
            \State $fronteraActual \gets$ Frontera($conjNodos$)
            \If {$fronteraActual > fronteraMax$}
                \State $fronteraMax \gets fronteraActual$
                \State $solucion \gets conjNodos$
            \EndIf
        \EndIf

    \Else
        \State GenerarSubconjuntos($conjNodos$, $actual + 1$)
        \State GenerarSubconjuntos($conjNodos \cup \{actual\}$, $actual + 1$)
    \EndIf

\EndFunction


$ $\newline

\Function{EsClique}{$conjNodos$}
    \State completar
\EndFunction

$ $\newline

\Function{GenerarSubconjuntos}{$conjNodos$}
    \State completar
\EndFunction

\end{algorithmic}
\end{algorithm}



\subsection{Complejidad}
\todo[inline]{Aca un mini analisis de complejidad}
