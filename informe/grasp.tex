% !TEX root = ./informe.tex
\section{Grasp}

\subsection{Explicación}


\subsection{Pseudocódigo}

Las funciones greedy.resolver(), local.resolver() y Frontera() no son incluidas aquí por ser iguales a las incluidas previamente. Las complejidades son $O(n^5)$, $O(n^5)$ y $O(n^2)$ respectivamente. \\

Referencias de variables globales para el pseudocódigo:
\begin{itemize}
    \item $n$: La cantidad de nodos
    \item $solucion$: Secuencia que contiene la clique solución
\end{itemize}

\begin{algorithm}[H]
\begin{algorithmic}
\Function{Resolver}{$maximo$}         
	\State $fronteraMax \gets 0$
	\State $fronteraNueva \gets 0$
	\State $repes \gets 0$
	\While{$repes < maximo$}
        \State $semilla \gets rand().mod(n)$
        \State $actual \gets greedy.resolver(semilla)$
    	\State $nueva \gets local.resolver(actual)$
    	\State $fronteraNueva \gets Frontera(nueva)$
    	\If {$fronteraNueva > fronteraMax$}
    		\State $solucion \gets nueva$
    		\State $fronteraMax \gets fronteraNueva$
            \State $repes \gets 0$
    	\Else 
    		\State $repes \gets repes + 1$
    	\EndIf
    \EndWhile
    \State return $solucion$

\EndFunction
\end{algorithmic}
\end{algorithm}


\subsection{Complejidad}


\subsection{Optimalidad}


\subsection{Experimentación}

